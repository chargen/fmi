\documentclass [10pt]{article} 

\usepackage{latexsym}
\usepackage{amssymb}
\usepackage{epsfig} 
\usepackage{fullpage}
\usepackage{enumerate}
\usepackage{xspace}
\usepackage{listings}
\usepackage{xcolor}
\usepackage{courier}
\usepackage{amsmath}

\newcommand{\true}{true}
\newcommand{\false}{false}


\pagestyle{plain}
\bibliographystyle{plain}


\title{Formale Methoden der Informatik \\
 Computability and Complexity: Exercises }
\author{Sebastian Geiger\\$<1127054>$}
\date{SS 2013}


  \newtheorem{theorem}{Theorem}
  \newtheorem{lemma}[theorem]{Lemma}
  \newtheorem{corollary}[theorem]{Corollary}
  \newtheorem{proposition}[theorem]{Proposition}
  \newtheorem{conjecture}[theorem]{Conjecture}
  \newtheorem{definition}[theorem]{Definition}
  \newtheorem{example}[theorem]{Example}
  \newtheorem{remark}[theorem]{Remark}
  \newtheorem{exercise}[theorem]{Exercise}

  \newcommand{\ra}{\rightarrow}
  \newcommand{\Ra}{\Rightarrow}
  \newcommand{\La}{\Leftarrow}
  \newcommand{\la}{\leftarrow}
  \newcommand{\LR}{\Leftrightarrow}

  \renewcommand{\phi}{\varphi}
  \renewcommand{\theta}{\vartheta}


\newcommand{\ccfont}[1]{\protect\mathsf{#1}}
\newcommand{\NP}{\ccfont{NP}}

\newcommand{\NN}{\textbf{N}}
\newcommand{\IN}{\textbf{Z}}
\newcommand{\bigO}{\mathrm{O}}
\newcommand{\bigOmega}{\Omega}
\newcommand{\bigTheta}{\Theta}
\newcommand{\REACHABILITY}{\mbox{\bf REACHABILITY}}
\newcommand{\MAXFLOW}{\mbox{\bf MAX FLOW}}
\newcommand{\MAXFLOWD}{\mbox{\bf MAX FLOW(D)}}
\newcommand{\MAXFLOWSUB}{\mbox{\bf MAX FLOW SUBOPTIMAL}}
\newcommand{\MATCHING}{\mbox{\bf BIPARTITE MATCHING}}
\newcommand{\TSP}{\mbox{\bf TSP}}
\newcommand{\TSPD}{\mbox{\bf TSP(D)}}

\newcommand{\TwoSAT}{\mbox{\bf 2-SAT}\xspace}
\newcommand{\ThreeCol}{\mbox{\bf 3-COLORABILITY}}
\newcommand{\TwoCol}{\mbox{\bf 2-COLORABILITY}}
\newcommand{\OneCol}{\mbox{\bf 1-COLORABILITY}}
\newcommand{\kCol}{\mbox{\bf k-COLORABILITY}}
\newcommand{\HamPath}{\mbox{\bf HAMILTON-PATH}}
\newcommand{\HamCycle}{\mbox{\bf HAMILTON-CYCLE}}

\newcommand{\ONESAT}{\mbox{\bf 1-IN-3-SAT}}
\newcommand{\MONONESAT}{\mbox{\bf MONOTONE 1-IN-3-SAT}}
\newcommand{\kSAT}{\mbox{\bf k-SAT}}
\newcommand{\NAESAT}{\mbox{\bf NAESAT}}


\renewcommand{\labelenumi}{(\alph{enumi})}

%%% useful macros for Turing machines:
\newcommand{\blank}{\sqcup}
\newcommand{\ssym}{\triangleright}
\newcommand{\esym}{\triangleleft}
\newcommand{\halt}{\mbox{h}}
\newcommand{\yess}{\mbox{``yes''}}
\newcommand{\nos}{\mbox{``no''}}
\newcommand{\lmove}{\leftarrow}
\newcommand{\rmove}{\rightarrow}
\newcommand{\stay}{-}
\newcommand{\diverge}{\nearrow}
\newcommand{\yields}[1]{\stackrel{#1}{\rightarrow}}

\newcommand{\HALTING}{\mbox{\bf HALTING}}

\definecolor{lightlightgray}{gray}{0.9}
\definecolor{OliveGreen}{cmyk}{0.64,0,0.95,0.40}

\begin{document}

\lstset{
language=C,                             % Code langugage
basicstyle=\ttfamily,                   % Code font, Examples: \footnotesize, \ttfamily
keywordstyle=\color{OliveGreen},        % Keywords font ('*' = uppercase)
commentstyle=\color{gray},              % Comments font
numbers=left,                           % Line nums position
stepnumber=1,                           % Step between two line-numbers
numbersep=5pt,                          % How far are line-numbers from code
backgroundcolor=\color{lightlightgray}, % Choose background color
frame=none,                             % A frame around the code
tabsize=4,                              % Default tab size
captionpos=b,                           % Caption-position = bottom
breaklines=true,                        % Automatic line breaking?
breakatwhitespace=false,                % Automatic breaks only at whitespace?
showspaces=false,                       % Dont make spaces visible
showtabs=false,                         % Dont make tabls visible
morekeywords={__global__, __device__},  % CUDA specific keywords
}


\maketitle
\textbf{ The deadline for submitting your exercises for review is
  April 14. If you would like to receive the corrections before the
  exercise session on April 15, you should submit your solutions to TUWEL no
  later than  April 7.  }



\begin{exercise}
  By providing a reduction from the \textbf{HALTING} problem to
  \textbf{REACHABLE-CODE}, prove that \textbf{REACHABLE-CODE} is
  undecidable.
\end{exercise}

\medskip

\textbf{Solution:}
Let $(\Pi, I)$ be an arbitrary instance of \textbf{HALTING}. We build an instance $(\Pi', n)$ of \textbf{REACHABLE-CODE} by setting $n=3$ and constructing $\Pi'$ as follows:
\begin{lstlisting}[mathescape, language=C, numbers=left, basicstyle=\ttfamily, backgroundcolor=\color{lightlightgray},]
    Boolean $\Pi'$ (String S):
        $\Pi$(S)
        return true
\end{lstlisting}

\textbf{Proof:}\\
$\Rightarrow$: Assume $(\Pi, I)$ is a positive instance of \textbf{HALTING}. If we set $S=I$ then by construction of $\Pi'$ the program halts and returns true, because $\Pi$ halts on $I$ and the return statement follows immediately after the call to $\Pi$.\\
$\Leftarrow$: Assume $(\Pi', n)$ is a positive instance of \textbf{REACHABLE-CODE}. That means there is an input S for $\Pi'$ such that $\Pi'$ returns true, but then by the construction of $\Pi'$ it means that $\Pi$ must halt on input $S$, in which case it is a positive instance of \textbf{HALTING}.

\begin{exercise}
    By providing a semi-decision procedure, prove that \textbf{CORRECTNESS} is
    semi-decidable.
\end{exercise}

\medskip

\textbf{Solution:}
We define $\Pi$ as an interpreter that takes an instance of \textbf{CORRECTNESS} $(S, I_1, I_2)$ where S is a program and $I_1$ and $I_2$ are inputs. $\Pi$ does the following:
\begin{itemize}
 \item $\Pi$ runs the program S with input $I_1$
 \item If $S$ terminates and returns $I_2$ then $\Pi$ returns true.
 \item If $S$ terminates and does not return $I_2$ then $\Pi$ returns false.
\end{itemize}

\textbf{Proof:}\\
$\Rightarrow$:\\
If $(S, I_1, I_2)$ is a positive instance of \textbf{CORRECTNESS}, then S when run on $I_1$ will terminate and return $I_2$. By constuction of $\Pi$, the interpreter will return true.\\
$\Leftarrow$:\\
If $(S, I_1, I_2)$ is a negative instance of \textbf{CORRECTNESS}, there are two cases:
\begin{itemize}
 \item If S does terminate on input $I_1$, but does not return $I_2$, then by definition of the interpreter, $\Pi$ will return false.
 \item If S does not terminate on input $I_1$, then the interpreter will run for ever, which is a correct behaviour for a semi-decision procedure.
\end{itemize}


\begin{exercise}
    By providing a reduction from \textbf{CORRECTNESS} to \textbf{HALTING},
    prove that \textbf{CORRECTNESS} is semi-decidable.
\end{exercise}

\medskip

\textbf{Solution:}
Let $(\Pi, I_1, I_2)$ be an arbitrary instance of \textbf{CORRECTNESS}. We construct an instance $(\Pi', S)$ of \textbf{HALTING} by setting $S=I_1$ and constructing $\Pi'$ as follows:
\begin{lstlisting}[mathescape, language=C, numbers=left, basicstyle=\ttfamily, backgroundcolor=\color{lightlightgray},]
    Boolean $\Pi'$ (String S) {
        if ($	\Pi$ (S) = $I_2$) return true;
        else return false;
\end{lstlisting}

\textbf{Proof}:\\
$\Rightarrow$:\\
Assume $(\Pi, I_1, I_2)$ is a positive instance of \textbf{CORRECTNESS}. Then by the construction of $\Pi'$, $\Pi$ halts on S and returns $I_2$, in which case $\Pi'$ will return true, that means it is a positive instance of \textbf{HALTING}.\\
$\Leftarrow$:\\
Assume we have a positive instance ($\Pi', S)$. That means that $\Pi'$ terminates and returns true. In that case $\Pi$ when run on input $S$ terminates and returns $I_2$, in which case it is a positive instance of \textbf{CORRECTNESS}.

We know that \textbf{HALTING} is semi-decidable. Then by definition of the reduction, \textbf{CORRECTNESS} is semi-decidable.

\begin{exercise}
    Prove that the following problem is undecidable:

  \begin{center}
    \fbox{
      \begin{minipage}[c]{.9\linewidth}
        \textbf{ALL-FALSE}

\medskip INSTANCE: A program $\Pi$ that takes as input a natural number and
returns \emph{true} or \emph{false}. It is guaranteed that $\Pi$ terminates on
any input.
 

\medskip

QUESTION: Is it the case that  $\Pi(k)=false$ for \emph{all} natural numbers $k$?
      \end{minipage}
    }
  \end{center}

  Hint: For your proof you may assume the availability of an interpreter for
  instances of \textbf{HALTING}. In particular, you have available a decision
  procedure $\Pi_{int}$ that does the following:
  \begin{enumerate}\item $\Pi_{int}$ takes as input a program $\Pi$,  a
    string $I$, and a natural number $n$.

  \item $\Pi_{int}$ emulates the first $n$ steps of the run of $\Pi$ on
    $I$. If $\Pi$ terminates on $I$ within $n$ steps, then $\Pi_{int}$ returns
    \emph{true}. Otherwise, $\Pi_{int}$ returns \emph{false}.
  \end{enumerate}

    
\end{exercise}

\medskip

\textbf{Solution:} your solution


\medskip

  Suppose you have $n$ processes, where some processes may need to
  communicate with each other. Suppose you also have $m$ computers,
  where some of them are connected by a (fast) direct network
  connection. Each computer has a limit on the number of processes it
  can run. Your problem is to assign processes to computers so that
  the limits are obeyed and all the processes that need to communicate
  can communicate. This can be formalized as follows:

  \begin{center}
    \fbox{
      \begin{minipage}[c]{.9\linewidth}
        \textbf{ASSIGNMENT}

        \medskip INSTANCE: A pair $G_1=(V_1,E_1)$ and $G_2=(V_2,E_2)$
        of undirected graphs, and a function $limit$ that assigns to
        each $v\in V_2$ an integer. The graph $G_2$ is reflexive,
        i.e. for every $v\in V_2$, $[v,v]\in E_2$.

        \medskip

        QUESTION: Does there exist an assignment $\mu$ that assigns to
        each vertex in $V_1$ a vertex in $V_2$ such that:

        \begin{enumerate}[(A)]
        \item if $[v,v']\in E_1$, then also $[\mu(v),\mu(v')]\in E_2$, and
        \item for every vertex $v$ in $V_2$, no more than $limit(v)$
          nodes of $V_1$ are assigned to $v$.
        \end{enumerate}

      \end{minipage}
    }
  \end{center}





\begin{exercise}
    Give a proof that \textbf{ASSIGNMENT} is in $\NP$, i.e.\,define a
    certificate relation and briefly discuss that it is polynomially
    balanced and polynomial-time decidable.
\end{exercise}

\medskip

\textbf{Solution:}
We define a certificate relation for \textbf{ASSIGNMENT} as follows:\\
\begin{align}
 R &= \{ ((G_1, G_2, limit), \mu) | \mu \text{ is a correct assignment that satisfies the conditions A and B} \}
\end{align}

R is a certificate relation for \textbf{ASSIGNMENT}, because an instance $(G_1, G_2, limit)$ is a positive instance of \textbf{ASSIGNMENT} $\Leftrightarrow$ there exists an assignment $\mu$, such that the conditions A and B of \textbf{ASSIGNMENT} are satisfied  $ \Leftrightarrow ((G_1, G_2, limit), \mu) \in R$.\\

R is polynomially balanced, because we only require space that is linear in the size of $V_1$.

R is polynomially time-decidable, because we can check condition A in polynmial time in the size of $E_1$ and $E_2$ and condition B in polynomial time in the size of $(V_2)$ satisfies the conditions A and B.


\begin{exercise}
    Define a polynomial-time reduction from \textbf{CLIQUE} to
    \textbf{ASSIGNMENT}.
\end{exercise}
Note: the result of Exercise 5 together with the reduction show that
\textbf{ASSIGNMENT} is $\NP$-complete.

\medskip

\textbf{Solution:} your solution

\medskip
    We consider a polynomial-time reduction from \textbf{INDEPENDENT SET} to
    \textbf{SAT}.  Let an arbitrary instance of \textbf{INDEPENDENT SET} be
    given by the undirected graph $G=(V,E)$ and integer $k$. Let $V$ be of the
    form $V=\{b_1,\ldots,b_m\}$.  We construct a propositional formula
    $\phi_{G,k}$ (i.e. an instance of \textbf{SAT}) as follows.
First of all, we use the following propositional variables:
\begin{enumerate}[-]
\item $M_{i,b_j}$ for each $1 \leq i \leq k$ and $ 1 \leq j \leq m$
(intended meaning: $M_{i,b_j}$ is set to $\true$ in a model of $\phi_{G,k}$ if and only if the number $i$ is
    assigned to the node $b_j$ of $G$).
  \end{enumerate} 

 Then the formula $\phi_{G,k}$ is defined as $\phi_{G,k}=
  \alpha_1\land \alpha_2 \land \alpha_3 $, where

\[\alpha_1=\bigwedge_{1 \leq i \leq k} \big ( \bigvee_{1 \leq j \leq m}
    M_{i,b_j} \big )\]

    \[\alpha_2=\bigwedge
    \limits_{ (1 \leq n \leq m) \wedge (1 \leq i,j \leq k)\wedge (i\neq j) }
    \neg (M_{i,b_n} \land M_{j,b_n})\]

  \[\alpha_3=\bigwedge_{[v_1,v_2] \in E } \bigwedge_{ 1 \leq i,j \leq k } \neg(
  M_{i,v_1}\land M_{j,v_2} ) \]

{\em Informal explanation of the reduction.}
Intuitively (!), the formulae $\alpha_1,\alpha_2,\alpha_3$
can be explained as follows. 
%
\begin{itemize}
% \item The formula $\alpha_1$ simply ``encodes'' the edges of $G$ as
%     a logical formula.
\item The formula $\alpha_1$ expresses the condition that each number $i\in
  \{1,\ldots k,\}$ must be assigned to at least one node from $G$.
\item The formula $\alpha_2$ makes sure that a pair of numbers do not share
  the same vertex in $G$. Thus $\alpha_1$ together with $\alpha_2$ make sure
  that at least $k$ nodes from $G$ have numbers ``assigned''.

\item The formula $\alpha_3$ ensures that for every edge $[v_1,v_2]$ of $G$ we
  have not assigned numbers to both $v_1$ and $v_2$.
\end{itemize}

\noindent {\em Remark.} All the above comments are {\em explanations\/} of the
intuition of the problem reduction. They are {\em not proofs\/}!! When you are
requested to prove the correctness of the problem reduction, you are not
allowed to refer to these explanations. {\em Your proofs have to be
  self-contained\/}!

\begin{exercise}
  Prove formally the ``$\Rightarrow$'' direction of the correctness of
  the reduction, i.e.  prove the following statement: if $G$ has an
  independent set $I$ of size $\geq k$, then there exists a truth
  assignment $T$ that makes $\phi_{G,k}$ evaluate to $\true$.
\end{exercise}

\medskip

\textbf{Solution:} Let (G, k) be a positive instance of \textbf{INDEPENDENT-SET}, that means there exists an $I \subset G$ with $|I| \geq k$, which satisfies the conditions of \textbf{INDEPENDENT-SET}. Without loss of generality, we assume $I$ is of the form $I=\{b_1, ..., b_k\}$.
For every $i \in K$ where $K$ is of the form $K=\{1, ..., k\}$ we set $M_{i,b_j} \in T$ to true, if $b_j$ is the i'th node in I, that is $j=i$.\\

Because I is of size greater or equal to k, it follows that for every i, there is must be an $M_{i,b_j}$ which is true. Therefore $\alpha_1$ must be true.

Assume $\alpha_2$ is false. Then there must be two arbitrary variables $M_{i_1, b_j}, M_{i_2, b_j}$ of $T$ with $i_1 \neq i_2$ which are both true. Then it follows that $b_j = i_1$ and $b_j = i_2$ which means $i_1 = i_2$, but this is a contradiction to the definition of $\alpha_2$. Thus $\alpha_2$ must be true under the above assignment.

Assume $\alpha_3$ is false under the above assignment. Then we can find two $M_{i,b_j}$ which are both true. Then there exist two nodes $v_1, v_2 \in V$ such that $[v_1, v_2] \in E$ and $v_1, v_2 \in I$. This is a contradiction. But then $\alpha_3$ must be true under the above assignment.

We have shown that all three subformulas evaluate to true under the above assignment. then it follows that $\phi_{G, k}$ evaluates to true.

\begin{exercise}
  Prove the ``$\Leftarrow$'' direction of the correctness of the
  reduction, i.e.  prove the following statement: if $\phi_{G,k}$ is
  satisfiable, then there exists some independent set $I$ in $G$ of
  size $\geq k$.
\end{exercise}

\medskip

\textbf{Solution:} Your Solution.

\begin{exercise}
    Argue that the following problem is solvable in logarithmic space:

  \begin{center}
    \fbox{
      \begin{minipage}[c]{.9\linewidth}
        \textbf{SAME-DIGITS}

        \medskip INSTANCE: A pair $L_1,L_2$ of lists, where each list
        contains some digits from $0,\ldots,9$.
 

\medskip

QUESTION: Is the set of digits occurring in $L_1$ equal to the set of digits
occurring in $L_2$?
      \end{minipage}
    }
  \end{center}
    

\end{exercise}

\medskip

\textbf{Solution:} your solution

\begin{exercise}
    Let $L=\{w\in \{1\}^{*}\mid w \mbox{ has length }3k \mbox{ for some
      integer }k\geq 0\}$, i.e. $L$ is the set of all strings $w$ such that (a) $w$
    is built using the symbol $1$, and (b) the length of $w$ is a multiple of
    $3$. Define a Turing machine $M$ that decides $L$, i.e. define a tuple $M=
    (K,\Sigma, \delta, s)$ such that, for all $w\in \{1\}^{*}$, we have:
  \begin{itemize}
  \item if $w\in L$, then $M(w)="yes"$;
  \item if $w\not \in L$, then $M(w)="no"$.
  \end{itemize}
 Additionally, provide a high-level description of $M$.    
\end{exercise}

\medskip

\textbf{Solution:} your solution



\end{document}


